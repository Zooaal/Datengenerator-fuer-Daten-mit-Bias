\chapter{Einleitung}
    \begin{onehalfspace}    
        \label{sec:einleitung}
        Die fortschreitende Digitalisierung ist kaum noch aus unserem Alltag wegzudenken. Durch immer mehr Programme, die einem den Alltag erleichtern sollen, nutzen wir die Errungenschaften der Digitalisierung täglich. Häufig ist hier die Rede von künstlicher Intelligenz. Dabei ist uns meist nicht einmal Bewusst, dass im Hintergrund mit künstlicher Intelligenz gearbeitet wird. Egal ob als intelligenten Routenplaner oder Sprachsteuerung, hinter all diese Anwendung steckt heute nicht mehr nur ein Optimierungsalgorithmus sondern \ac{KI}. \\
        Mit der Digitalisierung hat man begonnen große Datenmengen zu sammeln. Durch den technischen Fortschritt im Bereich von Big Data, werden diese Datenmengen heutzutage unvorstellbar groß. Mit dem Erfassen und Speichern von Daten ist man in der Lage seine Produkte stetig zu verbessern und sogar neue Geschäftsmodelle zu schaffen. Zu diesen neuen Geschäftsmodellen gehört auch die nicht mehr aus unserem Alltag wegzudenkenden \ac{KI}. Sie ermöglicht es uns Entscheidungen zu treffen, wie sie auch ein Mensch treffen könnte, aber auch Vorhersagen zu machen, was zwar für den Menschen möglich ist, aber mit viel Aufwand verbunden ist. Egal ob eine Entscheidung oder eine Vorhersage von einer \ac{KI} getroffen werden, dahinter stehen Daten die bereits gesammelt wurden und die Entscheidungsgrundlage für die \ac*{KI} bilden. Aus diesem Grund werden Daten eine wertvolle Ressource, sobald man anfängt die Daten zu verarbeitet und aktiv zu nutzen. \\
        Für \ac*{KI} werden die Daten zum Lernen benutzt. Entscheidend für die Qualität der \ac*{KI} ist daher in den meisten Fällen die Datengrundlage auf der die \ac*{KI} besiert. Lernen bedeutet, dass Zusammenhänge und die dadurch abgebildeten Verhaltensweisen von der \ac*{KI} erkannt und sich selbst angeeignet werden. Durch diese Art des Lernens, wie auch wir Menschen unser Wissen erlernen, ergeben sich nicht nur Potentiale sondern auch Gefahren! Abhängig von der Datenqualität und Richtigkeit \ac*{bzw} Zuverlässigkeit der Daten werden zukünftige Entscheidungen und Vorhersagen getroffen. Eine \ac*{KI} betrachtet dabei die Daten vollkommen neutral ohne Hintergrundwissen über Richtigkeit und Zuverlässigkeit. Deshalb können Verzerrungen in den Daten durch die \ac*{KI} nicht erkannt werden. Diese Verzerrung wird auch Bias gennant und befindet sich in den Trainingsdaten mit denen die \ac*{KI} lernt. Die Folge daraus ist, dass sich \ac*{KI}s benachteiligende und diskriminierende Verhaltensweisen angeeignen und diese selbst in der Praxis ausüben.\\
        Insbesondere für durch Computer getroffene Entscheidungen und Vorhersagen spielt die Ethik daher eine große Rolle. Diese kann in der Regel nicht aus den Daten erlernt werden und hängt von uns Nutzern ab. So stellt sich die Frage wie sollen Menschen mit Entscheidungen durch \ac*{KI} umgehen und sich auf diese Verlassen.
        
        \newpage

        \section{Motivation}
        \label{subsec:motivation}
        Eine \ac*{KI} und deren Entscheidungen basieren stets auf Daten aus der Vergangenhiet, den Trainingsdaten. Wenn diese Trainingsdaten durch einen Bias verzerrt sind, ist das nicht unbedingt bekannt. In den meisten Fällen ist eine solche Verzerrung verborgen und wird erst im produktiven Betrieb der \ac*{KI} festgestellt.
        
        Diese Verzerrungen führen dann häufig zu Skandalen in der Medienwelt. Es wurde bereits diverse Male in der Presse darüber berichtet, dass bspw. in Unternehmen Bewerbungen durch ein \ac*{KI} vorsortiert werden und eine Diskriminierung in dem Muster der Auswahl erkennbar waren. Diese Diskriminierungen sind jedoch nicht zu vergessen immer auf Daten zurückzuführen und somit auch auf die Ersteller der Daten, also die Menschen dahinter.

        -	Anonymisierung/Pseudonymisierung bei besonders gro{\ss}en Datensätzen ist schwierig \\
        -	Nachvollziehbarkeit von Bias verzerrten Daten \\
        -	Veranschaulichung von Bias in Daten für die Allgemeinheit, um auf das Problem im Bereich ML aufmerksam zu machen 

        \section{Zielsetzung}
        \label{subsec:zielsetzung}
        -	Datengenerator für Bias verzerrte Daten \\
        -	Visualisierung von Bias in Lerndaten für ML \\
        -	Gesamt Produkt zur Erstellung von Daten und derer Bias Visualisierung für die Lehre \\
        \\
        -   1 Satz, was sollen wir machen --> Stichwortliste mit Anforderungen \\
        \\
        -   Maschinelles Lernen hängt von den Trainingsdaten ab.\\
        -   Trainingsdaten können einen Bias Data enthalten.\\

        \section{Aufbau der Arbeit}
        \label{subsec:aufbau der arbeit}

        \newpage

    \end{onehalfspace}
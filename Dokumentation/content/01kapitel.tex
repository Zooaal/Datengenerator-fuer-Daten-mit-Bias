\chapter{Einleitung}
    \begin{onehalfspace}    
        \label{sec:einleitung}
        Die fortschreitende Digitalisierung ist kaum noch aus unserem Alltag wegzudenken. Durch immer mehr Programme, die einem den Alltag erleichtern sollen, nutzen wir die Errungenschaften der Digitalisierung täglich. Häufig ist hier die Rede von künstlicher Intelligenz. Dabei ist uns meist nicht einmal Bewusst, dass im Hintergrund mit künstlicher Intelligenz gearbeitet wird. Egal ob als intelligenten Routenplaner oder Sprachsteuerung, hinter all diese Anwendung steckt heute nicht mehr nur ein Optimierungsalgorithmus sondern \ac{KI}. 
        \\
        Mit der Digitalisierung hat man begonnen große Datenmengen zu sammeln. Durch den technischen Fortschritt im Bereich von Big Data, werden diese Datenmengen heutzutage unvorstellbar groß. Mit dem Erfassen und Speichern von Daten ist man in der Lage seine Produkte stetig zu verbessern und sogar neue Geschäftsmodelle zu schaffen. Zu diesen neuen Geschäftsmodellen gehört auch die nicht mehr aus unserem Alltag wegzudenkenden \ac{KI}. Sie ermöglicht es uns Entscheidungen zu treffen, wie sie auch ein Mensch treffen könnte, aber auch Vorhersagen zu machen, was zwar für den Menschen möglich ist, aber mit viel Aufwand verbunden ist. Egal ob eine Entscheidung oder eine Vorhersage von einer \ac{KI} getroffen werden, dahinter stehen Daten die bereits gesammelt wurden und die Entscheidungsgrundlage für die \ac*{KI} bilden. Aus diesem Grund werden Daten eine wertvolle Ressource, sobald man anfängt die Daten zu verarbeitet und aktiv zu nutzen. 
        \\
        Für \ac*{KI} werden die Daten zum Lernen benutzt. Entscheidend für die Qualität der \ac*{KI} ist daher in den meisten Fällen die Datengrundlage auf der die \ac*{KI} basiert. Lernen bedeutet, dass Zusammenhänge und die dadurch abgebildeten Verhaltensweisen von der \ac*{KI} erkannt und sich selbst angeeignet werden. Durch diese Art des Lernens, wie auch wir Menschen unser Wissen erlernen, ergeben sich nicht nur Potentiale sondern auch Gefahren! Abhängig von der Datenqualität und Richtigkeit \ac*{bzw} Zuverlässigkeit der Daten werden zukünftige Entscheidungen und Vorhersagen getroffen. Eine \ac*{KI} betrachtet dabei die Daten vollkommen neutral ohne Hintergrundwissen über Richtigkeit und Zuverlässigkeit. Deshalb können Verzerrungen in den Daten durch die \ac*{KI} nicht erkannt werden. Diese Verzerrung wird auch Bias genannt und befindet sich in den Trainingsdaten mit denen die \ac*{KI} lernt. Die Folge daraus ist, dass sich \ac*{KI}s benachteiligende und diskriminierende Verhaltensweisen aneignen und diese selbst in der Praxis ausüben.
        \
        Insbesondere für durch Computer getroffene Entscheidungen und Vorhersagen spielt die Ethik daher eine große Rolle. Diese kann in der Regel nicht aus den Daten erlernt werden und hängt von uns Nutzern ab. So stellt sich die Frage wie sollen Menschen mit Entscheidungen durch \ac*{KI} umgehen und sich auf diese Verlassen. Diese fehlende Ethik sorgt für nicht zu Vernachlässigende Verzerrungen und bildet so einen Bestandteil der "Dark side of KI". \ac{bzw}

        \section{Motivation}
        \label{subsec:motivation}
        Eine \ac*{KI} und deren Entscheidungen basieren stets auf Daten aus der Vergangenheit, genauer den Trainingsdaten. Wenn diese Trainingsdaten aufgrund unterschiedlichste Ursachen ein unerwünschtes Verhaltensmuster beinhalten spricht man häufig von Bias. So können zum Beispiel Entscheidungen aufgrund eines unbekannten Faktors, häufig Diskriminierung,  basieren. Die Problematik liegt darin, dass den Endnutzer in der Regel nicht bekannt ist, dass es einen Bias in den Daten geben kann und einer \ac*{KI} blind Vertrauen. In den meisten Fällen ist eine solche Verzerrung verborgen und wird erst im produktiven Betrieb der \ac*{KI} festgestellt. 
        \\
        Diese Verzerrungen führen dann häufig zu Skandalen in der Medienwelt. Es wurde bereits diverse Male in der Presse darüber berichtet, dass bspw. in Unternehmen Bewerbungen durch ein \ac*{KI} vorsortiert wurden und ein diskriminierendes Muster in den Trainingsdaten erkennbar war.
        \\
        Diese Diskriminierungen sind jedoch nicht zu vergessen immer auf Daten zurückzuführen und somit auch auf die Ersteller der Daten, also die Menschen dahinter.
        \\
        Die Motivation dieser Arbeit ergibt sich aus den zuvor beschriebenen Problemen. \ac*{KI} unterstützt uns in jeglichen Aufgaben in unserem Alltag.
        \\
        -	Anonymisierung/Pseudonymisierung bei besonders gro{\ss}en Datensätzen ist schwierig \\
        -	Nachvollziehbarkeit von Bias verzerrten Daten \\
        -	Veranschaulichung von Bias in Daten für die Allgemeinheit, um auf das Problem im Bereich ML aufmerksam zu machen 

        \section{Zielsetzung}
        \label{subsec:zielsetzung}
        Um das Bewusstsein zu schaffen, dass beim Einsatz von \ac*{KI} die Ethik eine große Rolle spielt, sollen verzerrte Trainingsdaten erzeugt werden. Dafür soll im Rahmen der Lehre ein Datengenerator geschaffen werden, der es ermöglicht verzerrte Daten zu erzeugen. Mit Hilfe dieser Daten können dann anhand zufällig erzeugter Daten mit festgelegtem Bias veranschaulicht werden, welche Auswirkungen dieser auf die \ac*{KI} und deren Entscheidungen hat. 
        \\
        Die Anforderungen, die sich daraus ergeben, sind:
        \begin{itemize}
            \item Die Konzeption für zwei Szenarien, die möglichst Realitätsnah sind und künstlich verzerrt werden können.
            \item Das Erstellen eines Datengenerator für Daten, die durch einen Bias verzerrt sind.
            \item Das Erstellen einer Auswertung für den erzeugten Datensatz, der den Bias für die Lehre veranschaulicht.
        \end{itemize} 
        Gesamt Produkt zur Erstellung von Daten und derer Bias Visualisierung für die Lehre \\
        1 Satz, was sollen wir machen --> Stichwortliste mit Anforderungen \\

        \section{Aufbau der Arbeit}
        \label{subsec:aufbau der arbeit}
        Der erste Abschnitt ist in drei Passagen aufgeteilt. Zu Beginn wird das allgemeine Thema der Daten als Grundlage für \ac*{ML} betrachtet. Dabei wird insbesondere auf die Datenqualität eingegangen. Des weiteren wird das Thema Bias, also die Verzerrung in den Daten, auf Basis der Literatur veranschaulicht. In der folgenden Passage wird auf \ac*{KI} und \ac*{ML} eingegangen. Ebenso wird die Ethik in der \ac*{KI} betrachtet. Die letzte Passage setzt sich dann mit Bias in \ac*{KI} Trainingsdaten auseinander. Dabei liegt der Fokus auf der dadurch möglicherweise entstehenden Diskriminierung aber im Gegensatz dazu auch mit Ansätzen und Konzepten von Gegenmaßnahmen.
        Im nächsten großen Abschnitt wird die praktische Umsetzung des Datengenerators näher betrachtet. Dafür werden zu Beginn die zwei Szenarien ausgearbeitet und näher beschrieben. Als nächstes werden die daraus entstehenden Anforderungen in Form eines Konzepts aufgestellt. Dieses unterscheidet sich in Fein und Grobkonzept und beschreibt die logischen Funktionen. Abschließend wird noch die konkrete Implementierung beschrieben.
        \\
        FEHLT HIER NICHT NOCH IWIE ERGEBNISSE \\
        UND ES FEHLT TABLEAU
        \\
        Abschließend werden alle Erkenntnisse gesammelt und zusammengefasst. Hier wird auch das Ergebnis der Arbeit kritisch Reflektiert und Evaluiert. Zum Schluss wird noch ein kurzer Ausblick darüber gegeben, welche Relevanz Bias in der \ac*{KI} zukünftig haben wird. 

        

        \newpage

    \end{onehalfspace}